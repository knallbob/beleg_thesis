\chapter{Introduction}
\pagenumbering{arabic}
In the past, the interaction between the operating system and the processor was
relatively straightforward. A privileged mode, known as the kernel mode, granted
access to privileged features such as page table manipulation and interrupt
handling. This mode hosted the operating system, overseeing applications in the
unprivileged user mode. However, in recent years, the operating systems and
security communities have sought more than these two simplistic modes to enhance
performance and security. In response to this demand, hardware manufacturers introduced a variety of
new modes, addressing aspects like nested paging for hypervisors or modes
designed for heightened isolation, providing an exceptionally secure context for
execution. While these modes offer solutions for various specialized use cases,
they also introduce significant complexity to hardware, incurring substantial
development costs. As ideas for new modes continue to emerge, a crucial question
arises: Is it feasible to abstract the intricate mode-handling hardware into
software, thereby enabling software-defined CPU modes?\par 
Constructing a prototype of a CPU equipped with software-defined modes poses two
significant challenges. Firstly, there is the issue of handling a mode switch in
software without incurring a significant performance overhead. Secondly, the design of
a memory protection mechanism needs to address the complexity of accommodating
an arbitrary number of modes. The current paging mechanism, designed with only two
modes in mind, is optimized for this specific use case and falls short of
providing the required performance to handle a substantial number of modes, due
to a inefficient handling of access rights for multiple modes which is done on a
three bit per mode level. Consequently, a novel approach to memory protection
must be devised, one that transcends the limitations of the existing mechanism and allows for the dynamic
management of an extensive range of modes. The resolution of these challenges is
pivotal in realizing a functional prototype that not only supports
software-defined CPU modes effectively but also addresses the associated
performance considerations and memory protection requirements.\par
Previous efforts, particularly the work of von Elm.\cite{Cve}, have delved into addressing
the two aforementioned challenges. Von Elm's work, while successfully
implementing a mode switch mechanism with multiple definable modes, placed a
greater emphasis on devising a memory protection mechanism suitable for
accommodating multiple modes. In focusing on this aspect, he developed a
highly efficient protection mechanism that leverages a combination of paging and
segmentation to meet the required specifications. However, it's essential to
note that the actual mode switch, despite the advancements in memory protection,
remains hardware-based in his implementation.To simplify the hardware
implementation, von Elm opted for a hierarchy-based approach, necessitating a
predefined supervisor mode. While effective in certain respects, this
hierarchical approach falls short of providing the flexibility inherent in a
fully software-based mode switch behavior.\par
In this study, I leverage von Elm's approach for memory protection but overhaul
the implementation of mode switching behavior, which is currently predominantly
hardware-based. I aim to make the switch behavior entirely definable in
software, introducing the concept of independent modes that can be dynamically
created and utilized during runtime. To achieve this, I introduce a novel mode
named the \emph{Mode Switch Mode}. This mode comes into play whenever a mode switch
is required and dictates the alterations to crucial registers necessary for the
transition. Adopting a design incorporating this mode empowers programmers to
write the code responsible for executing the actual mode switch. As an
alternative approach, the Mode Switch Mode is offloaded to a secondary CPU known
as the \emph{Control Core}. This dedicated CPU activates each time a mode switch
occurs and, based on its own programming, changes the mode of the main CPU. The
goal is to alleviate concerns about altering the CPU state for the code running
on the Control Core, thereby reducing the state-save overhead. While some
benchmarks demonstrate a performance gain, the limited scope of these
microbenchmarks and the absence of evaluation of hardware costs provide only
suggestive evidence rather than conclusive statements regarding the comparison
of the two approaches. Consequently, this thesis primarily underscores the
feasibility of software-defined CPU modes through prototypical implementations
of the two proposed designs.\par
This thesis is organized as follows: Chapter 2 provides an exploration of
preexisting mode isolation techniques, along with general information about CPU
architecture, with a focus on the Risc-V architecture, which serves as the
framework for my designs. The gem5 simulator is introduced as the primary tool
employed for the simulation of my designs. In Chapter 3, a comprehensive
overview of von Elm's work, which forms the basis of this thesis, is presented.
Chapter 4 delves into the detailed presentation of the two designs, discussing
various design choices. The subsequent chapter, Chapter 5, offers an insight
into the implementation of these designs within the gem5 simulator. Moving on to
Chapter 6, the microbenchmarks are explained, and the results of these
benchmarks are evaluated. Limitations of the benchmarks are also revealed in
this section. In Chapter 7, a more in-depth examination of these shortcomings is
provided, accompanied by an overview of potential avenues for future
improvements to mitigate these issues. Finally, Chapter 8 concludes the thesis,
offering a concise description of the contributions made in the course of this
research.