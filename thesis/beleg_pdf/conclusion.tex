\chapter{Conclusion}
In this thesis, I have explored the management of software-defined CPU modes
through two distinct designs. The central inquiry addressed whether it's viable
to make these modes entirely software-definable, departing from reliance on
pre-defined hardware. To this end, I transitioned the control plane into a
software environment, aiming to provide more flexibility and customization.\par
An earlier attempt by von Elm. \cite{Cve} primarily focused on memory protection, offering a
solution for fine-grained protection but with limitations in flexibility due to
hierarchical structuring and fixed supervisor modes. In response, my approach
extended definable permissions to include control and status registers, enabling
the creation of entirely software-definable modes, such as the supervisor mode.
Additionally, I introduced the Mode Switch Mode, executed entirely in software,
to assume hardware responsibilities, thereby eliminating hierarchical structures
and allowing arbitrary checks on mode transitions.\par
The subsequent design, the Control Core, proposed employing a dedicated CPU for
mode switches to potentially enhance system performance. Micro-benchmarks were
designed to evaluate functionality and performance, suggesting significant
performance enhancements with a second Core, particularly in scenarios utilizing
numerous registers. However, limitations in benchmark relevance to real-world
systems and unaddressed implementation costs highlight areas for further
investigation.\par
Despite these limitations, this thesis illustrates the feasibility of designing
a processor with freely definable modes, with the option of a dedicated CPU for
their management. This offers avenues for exploring alternative architectures
leveraging software-defined modes, paving the way for potential advancements in
processor design.\par